\documentclass{ltxdoc}
\usepackage{multicol}
\usepackage[tt=false, type1=true]{libertine}
\usepackage[varqu,scaled=.88]{zi4}
\usepackage[libertine]{newtxmath}
\usepackage[tableposition=top]{caption}
\usepackage{fancyvrb}
\usepackage{hypdoc}
\usepackage{enumitem}
\setlist%
{%
 topsep=0pt,%
 labelsep=6pt,
 noitemsep,%
 leftmargin=*
}
\setlist[description]{font=\normalfont\sffamily}

\def\file#1{\texttt{#1}}
\begin{document}
\title{\LaTeX\ class for AKA book  series}
\author%
    {%
      Linas Stonys\footnote{\href{mailto:lstonys@vtex.lt}{lstonys@vtex.lt}},
      \space
      Deimantas Gal\v{c}ius%
      \footnote{\href{mailto:deimantas.galcius@gmail.com}{\texttt{deimantas.galcius@gmail.com}}}
    }
\date{2017/08/02, v0.1}
\maketitle

\abstract%
    {%
      The package provides a class for typesetting books
      to be published with AKA books series.
    }

\tableofcontents

\section{Introduction}

The document class is built on \file{book.cls} class and requires the following packages:

\begin{multicols}{2}
\begin{itemize}
\item \file{mathptmx}
\item \file{graphicx}
\item \file{index}
\item \file{multicol}
\end{itemize}
\end{multicols}


%\end{document}

\section{Installation}

The latest version of the package can be found on GitHub: 
\url{https://github.com/vtex-soft/texsupport.aka}.

A bug report can be filed at \url{https://github.com/vtex-soft/texsupport.aka/issues}.
%At this address you can file a bug report—or even
%contribute your own enhancement making a pull request.

Most users should not attempt to install this package themselves, and rather rely on
their \TeX\ distributions to provide it. If you decide to install the package yourself, follow
the standard rules:
\begin{enumerate}
\item Put the file \file{akabook.cls}  to the places where \LaTeX\ 
can find them (see \cite{ref:ukfaq} or the documentation for your \TeX\ system).
\item Update the database of file names. Again, see \cite{ref:ukfaq} or the documentation for your
\TeX\ system for the system-specific details.
\end{enumerate}


The installation is optional and you can skip this phase.
The bundle is self-contained and after unzipping your have everything you need for a book preparation. 


\section{Book structure}\label{bookstructure}

Put each chapter of a book in separate files and load them in appropriate places
of main file. These places are marked with commands \verb|\frontmatter|, \verb|\mainmatter| and \verb|\backmatter|.

Place image files into  \file{img/} subfolder. 
The \LaTeX\ class file resides in  a dedicated \file{sty/} folder.



\section{Usage}

The class should be loaded with the following command:

\begin{verbatim}
\documentclass[<options>]{akabook}
\end{verbatim}

Options are available same as for \file{book} class.

\section{Single chapter}

Start every chapter in a new \texttt{tex} file and include it in your main file
with \verb|\include{}|.
A typical chapter coding is shown below:

\begin{Verbatim}
\chapter{Chapter Title\footnote{This is chapter footnote}}%
\section{...}
...
\subsection{...}
...
\section{..}
...
\end{Verbatim}

\section{Section headings}

There are four section head levels defined. Coding for different heading levels are shown below:
\begin{Verbatim}
\section{Head Level 1}
\subsection{Head Level 2}
\subsubsection{Head Level 3}
\paragraph{Head Level 4}
\end{Verbatim}


\section{Lists}

The \file{akabook.cls} uses standard LaTeX list environments \texttt{itemize} and \texttt{enumerate}.
\begin{Verbatim}
\begin{itemize}
\item ...
\item ...
\end{itemize}
\begin{enumerate}
\item ...
\item ...
\end{enumerate}
\end{Verbatim}


\section{Tables and figures}

Figures may be included using the command \verb!\includegraphics!. 
Use EPS file format for figures  working with LaTeX, and PDF, PNG, MPS file formats for pdfLaTeX. 
Do not use file extensions and path in order to load file. Figure mark up is as follows:
\begin{Verbatim}
\begin{figure}
\includegraphics{file-name}% no path, no extension
\caption{Figure caption}\label{fig:f01}
\end{figure}
\end{Verbatim}
Table environment may be enhanced depending on the model chosen. 
\begin{Verbatim}
\begin{table}
\caption{Table caption}
\label{tab:1}       % Give a unique label
\begin{tabular}{lll}
\hline
first & second & third  \\
\hline
number & number & number \\
number & number & number \\
\hline
\end{tabular}
\end{table}
\end{Verbatim}


\section{Theorems and alike environments}

It is recomended to use \verb!amsthm! package \cite{ref:amsthm} to make it easier 
to define theorem environments and the alike.
\begin{Verbatim}
\usepackage{amsthm}
\newtheorem{theorem}{Theorem}
\theoremstyle{definition}
\newtheorem{definition}{Definition}
\theoremstyle{remark}
\newtheorem{remark}{Remark}
\begin{theorem}[Optional title]\label{thm:01}
...
\end{theorem}
\end{Verbatim}


\section{Display mathematics}

AMS math coding is preferred for display mathematics \cite{ref:amsmath}.
Avoid \verb!eqnarray! environment for coding.


\section{Cross-references}

Cross-referencing is possible in \LaTeX\ for section headings, formulae, figure, tables, 
literature references, etc. For example, the words `Fig. 1' will never be more than simple 
text, whereas the proper cross-reference \verb!Fig.~\ref{fig:tiger}! may be turned into a 
hyperlink to the figure. In the same way, the words `Ref. [1]' will fail to turn into a 
hyperlink; the proper cross-reference is \verb!\cite{Knuth96}!.


\section{Bibliography}

For bibliography citation management it is recomended to use \file{natbib},
which 
is the most commonly used package for handling references in LaTeX.
You can choose between author--year (default) and numerical (option \verb!numbers!) citations.
Further customization 
can be made via \verb!\setcitestyle! macro (see \cite{ref:natbib}) for details.

\section{Index}

Index is an alphabetical list of words and 
expressions with the pages of the book upon which they are to be found. 
LaTeX supports the creation of indices with its package \file{index} (loaded automatically in \file{akabook}) \cite{ref:index}, 
and its support program \file{makeindex}, called on some systems \file{makeidx}.

%\section{Acknowledgement}

\section{Appendices}

An appendix, in a book, is a collection of extra or supplementary material 
generally used in books and academic writing and appears at the end of a book.
 
%\section{Glossary List}


% \newpage
% \section{Submission}
%
% Submit one single file as a zip archive. 
% Pack your root folder \verb!<your-project-name>! with files and subfolders. 
% Check that subfolders \file{sty/}, \file{img/}, or \file{chapterNN/} (if any) are 
% present in a zip file.

\begin{thebibliography}{9}
\bibitem{ref:ukfaq} UK \TeX\ Users Group. UK list of \TeX\ frequently asked questions. 
\url{http://www.tex.ac.uk}, 2016.

\bibitem{ref:mdframed} 
M.~Daniel, E.~Schubert. \textit{The mdframed package}, 
2013. \url{http://www.ctan.org/pkg/mdframed}.

\bibitem{ref:amsmath}
F.~Mittelbach, R. Sch\"opf, M. Downes, D.M.~Jones, D.~Carlisle. \textit{AMS mathematical facilities for \LaTeX}, 216. \url{http://www.ctan.org/pkg/amsmath}. 

\bibitem{ref:amsthm}
American Mathematical Society. \textit{Typesetting theorems (AMS style)}, April 2015. 
\url{http://www.ctan.org/pkg/amsthm}.

\bibitem{ref:enumitem}
J. Bezos. 
\textit{Customizing lists with the enumitem package}, 2011. 
\url{http://www.ctan.org/pkg/enumitem}.

\bibitem{ref:natbib}
P.W. Daly.
\textit{Natural Sciences Citations and References}, 
2010. 
\url{https://www.ctan.org/pkg/natbib}.

\bibitem{ref:index}
D. M. Jones.
\textit{index – Extended index for LaTeX including multiple indexes}, 2004.
\url{https://www.ctan.org/pkg/index}.

\end{thebibliography}

\end{document}



%\subsection{titlepage}
%\subsection{Preface}
%\subsection{About Author}
%\subsection{Contributors Lists}
%\subsection{Toc, lof, lot, etc}
%\subsection{Dedication}



