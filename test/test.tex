\RequirePackage{etoolbox}
\csdef{input@path}{{../sty/}}

\documentclass{akabook}
\usepackage{amsmath}
\usepackage{amsthm}
\usepackage{lipsum}
\newtheorem{theorem}{Theorem}
\theoremstyle{definition}
\newtheorem{definition}{Definition}

\begin{document}
\chapter{Heading 1 (Introduction)}

\lipsum[1]
\section{Heading 2}
Long text long text long text long text long text long text
Long text long text long text long text long text long text\footnote{Corresponding Author, Corresponding author, Book Department, IOS Press, Nieuwe Hemweg
6B, 1013 BG Amsterdam, The Netherlands; E-mail: bookproduction@iospress.nl.}

\index{test a}
\index{test b}
\index{my test a}
\index{my test b}
\index{second}
\index{third}
\index{fourth}


\subsection{itemize}
 \begin{itemize}
   \item First level item long text long text long text long text long text long text long text long text long text long text
   \item First level item
   \begin{itemize}
     \item Second level item long text long text long text long text long text long text long text 
long text long text long text
     \item Second level item
     \begin{itemize}
       \item Third level item long text long text long text long text long text long text long text 
long text long text long text
       \item Third level item
       \begin{itemize}
         \item Fourth level item long text long text long text long text long text long text long 
text long text long text long text
         \item Fourth level item
       \end{itemize}
     \end{itemize}
   \end{itemize}
 \end{itemize}

Long text long text long text long text long text long text
Long text long text long text long text long text long text

%\tracingmacros=2
\subsection{itemize and enumerate}
\begin{enumerate}
   \item The labels consists of sequential numbers.
   \begin{itemize}
     \item here is second level item.
     \item The text in the entries may be of any length.
       \begin{itemize}
         \item here is third level item.
         \item The text in the entries may be of any length.
       \end{itemize}
   \end{itemize}
   \item The numbers starts at 1 with every call to the enumerate environment.
\end{enumerate}

Long text long text long text long text long text long text
Long text long text long text long text long text long text

\subsection{enumerate}
 \begin{enumerate}
   \item First level item  long text long text long text long text long text long text long text long text long text long text 
   \item First level item
   \begin{enumerate}
     \item Second level item long text long text long text long text long text long text long text 
long text long text long text
     \item Second level item
     \begin{enumerate}
       \item Third level item long text long text long text long text long text long text long text 
long text long text long text
       \item Third level item
       \begin{enumerate}
         \item Fourth level item long text long text long text long text long text long text long 
text long text long text long text
         \item Fourth level item
       \end{enumerate}
     \end{enumerate}
   \end{enumerate}
 \end{enumerate}

Items.png

\subsection{Heading 3}
Long text long text long text long text long text long text
Long text long text long text long text long text long text
\subsection{Heading 3}
Long text long text long text long text long text long text
Long text long text long text long text long text long text
\begin{quote}
Long text long text long text long text long text long text
Long text long text long text long text long text long text
\end{quote}
Long text long text long text long text long text long text
Long text long text long text long text long text long text

\subsubsection{Heading 4}
Long text long text long text long text long text long text
Long text long text long text long text long text long text

Long text long text long text long text long text long text
Long text long text long text long text long text long text

\paragraph{Heading 5}
Long text long text long text long text long text long text
Long text long text long text long text long text long text

\section{Heading 2 Long text long text long text long text long  text long text long text long text long text long text}

Long text long text long text long text long text long text
Long text long text long text long text long text long text

\section*{Heading 2 (unnumbered)}
Long text long text long text long text long text long text
Long text long text long text long text long text long text
\index{second}
\index{third}
\index{fourth}


\chapter{Heading 1}

\section{Introduction}
\label{intro}
Your text comes here. Separate text sections with
\section{Section title}
\label{sec:1}
Text with citations \cite{RefB} and \cite{RefJ}.
\subsection{Subsection title}
\label{sec:2}
as required. Don't forget to give each section
and subsection a unique label (see Sect.~\ref{sec:1}).
\paragraph{Paragraph headings} Use paragraph headings as needed.
\begin{equation}
a^2+b^2=c^2
\end{equation}

\begin{eqnarray}
A&=&B,\\
C&=&D,\\
E&=&F
\end{eqnarray}

\begin{theorem}
This is a standard theorem.
Long text long text long text long text long text long text
Long text long text long text long text long text long text
\end{theorem}

\begin{definition}[here is note]
This is a standard definition.
Long text long text long text long text long text long text
Long text long text long text long text long text long text
\end{definition}

\begin{figure}
\fbox{\makebox[50pt]{\vrule width0pt height50pt box}}
\caption{figure sample 
Long text long text long text long text long text long text
Long text long text long text long text long text long text}
\end{figure}



\tracingmacros=2
%\tracingmacros=0


% For tables use
\begin{table}
% table caption is above the table
\caption{Please write your table caption here}
\label{tab:1}       % Give a unique label
% For LaTeX tables use
\begin{tabular}{lll}
\hline\noalign{\smallskip}
first & second & third  \\
\noalign{\smallskip}\hline\noalign{\smallskip}
number & number & number \\
number & number & number \\
\noalign{\smallskip}\hline
\end{tabular}
\end{table}

\begin{table}
% table caption is above the table
\caption{Please write your table caption here}
\label{tab:1}       % Give a unique label
% For LaTeX tables use
\begin{tabular}{lll}
\hline
first & second & third  \\
\hline
number & number & number \\
number & number & number \\
\hline
\end{tabular}
\end{table}



\begin{appendix}
\chapter{Heading 1}

\section{Introduction}
Your text comes here. Separate text sections with
\end{appendix}

% Non-BibTeX users please use
\begin{thebibliography}{}
%
% and use \bibitem to create references. Consult the Instructions
% for authors for reference list style.
%

\bibitem{RefJ}
% Format for Journal Reference
A.N. Author, Article title, Journal Title 66 (1993), 856--890.
Long text long text long text long text long text long text
% Format for books
\bibitem{RefB}
A.N. Author, Book Title, Publisher Name, Publisher Location, 1995.
% etc
\end{thebibliography}

\printindex

\tableofcontents
\end{document}
